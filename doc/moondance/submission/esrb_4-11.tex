\documentclass[12pt]{article}

\usepackage{fullpage,times}

\begin{document}                            



\title{ESRB Qualitative Information}
\maketitle

{\bf Game:} Transcend

{\bf Author:} Jason Rohrer

\setcounter{section}{3}
\section{OVERALL GAME CONTEXT}

Transcend is an abstract shooting game with geometric, morphing graphics.
There are three levels.  In each level, the player guides a ship-like glyph 
around a grid.  Minor anti-glyphs (enemy objects) attack the player's glyph 
by shooting at it.  When the player's glyph is hit by an enemy projectile, it 
is knocked toward the center of the grid.

The player can increase the power of her projectiles by gathering elements
from around the grid and bringing them to the center.  The elements form
a visual collage, and each element also represents a section of music.
Thus, while powering up her projectiles, the player is also arranging a piece
of music that plays in the background.  Anti-glyphs also attack the player's 
collage by shooting at it, and these attacks push elements away from the 
center.

The player can defend her glyph and collage against enemy attacks by firing projectiles at the minor enemies.
These weak enemies can be destroyed with a single shot.   

The player's goal in each level is to destroy the major anti-glyph, a large
enemy object that circles around the outskirts of the grid.  When the player
approaches the major anti-glyph, the anti-glyph attacks the player's glyph
with powerful projectiles.  The player must shoot the major anti-glyph many
times in succession to destroy it.  After the major anti-glyph has been
destroyed, a portal opens that allows the player to move on to the next level.


\section{VIOLENCE}

\subsection{Violence Depicted}

{\bf Describe specific instances of the above forms of violence, and the context in which they occur:}
\\
\\
The player shoots projectiles at abstract enemy entities.  These entities are
controlled by the computer and behave autonomously, so they are best described 
as ``robots'' and not as ``objects.''  However, they do not represent robots 
specifically, but instead represent an abstract enemy force.  When enemy 
entities are destroyed, they morph into geometrical explosion displays and 
then fade away.
\\
\\
{\bf To what extent can the player control the depictions of violence?}
\\
\\
The player hits the SPACE bar to fire each projectile and aims projectiles
using the ARROW keys.  Thus, the player completely controls each act of 
violence.  However, the depiction of the destruction (in other words, the
explosion graphics that are displayed in each case) are controlled by the
computer.
\\
\\
{\bf Describe how violent acts are rewarded in the game, i.e., is the player 
  rewarded for successfully completing, avoiding, or preventing acts of 
  violence, aggression, injury, or death?  Include specific instances of such 
  rewards, and the context in which they occur:}
\\
\\
The player is rewarded indirectly for destroying minor anti-glyphs because the
player gets a break from enemy attacks until the next wave of anti-glyphs 
appears.  In other words, it is difficult to build an element collage while
anti-glyphs are attacking, so the player will benefit from destroying these minor enemies.

The player is rewarded directly for destroying the major anti-glyph by gaining
access to the next level.


\subsection{Violent Sound Effects}

{\bf Describe specific instances of these sound effects, and the context in which they occur:}
\\
\\
Each projectile fired is accompanied by a musical note.
Each enemy explosion is also accompanied by a musical sound.
These sound effects are abstract and do not directly represent the violent acts that they accompany.

\end{document}   
